%%%%%%%%%%%%%%%%%%%%%%%%%%%%%%%%%%%%%%%%%%%%%%%%%%%%%%%%%%
\section{\label{sec:Introduction}Introduction}
%%%%%%%%%%%%%%%%%%%%%%%%%%%%%%%%%%%%%%%%%%%%%%%%%%%%%%%%%%

Invoking something invisible that even our most sensitive detectors cannot detect does not sound like science. Yet, in contemporary cosmology, we have so much indirect evidence for Dark Matter (DM) that hardly any scientist doubts its existence. DM is a non-luminous form of matter, i.e., it does not interact with light -- unlike everyday objects around us. Nevertheless, having entered an era of precision cosmology, we have been able to determine that DM outweighs ordinary matter by a factor of about five in the energy balance of the Universe~\cite{Planck:2015xua}. Further observational evidence such as galaxy~\cite{Begeman:1991iy} or cluster dynamics~\cite{COMA} and the Bullet Cluster~\cite{Clowe:2006eq} allow us to constrain the properties of DM: we are sure that it is a form of matter (and not, e.g., modified gravity~\cite{Lage:2014yxa}), and our best guess for its identity is a new elementary particle that is electrically neutral and massive~\cite{Bertone:2004pz}. Importantly, it must have been produced in the early Universe in the right amounts and with a momentum spectrum suitable not to spoil the formation of cosmic structures.

It is generally accepted that DM was responsible for the emergence of structures in the Universe~\cite{Primack:1997av}, as seen in elaborate $N$-body simulations on high performance computers~\cite{Springel:2005nw}. Historically, the most generic DM candidates were WIMPs (Weakly Interacting Massive Particles), which appear in popular scenarios like supersymmetry. Such particles typically form \emph{cold} DM (CDM), i.e., particles produced by thermal freeze-out~\cite{Lee:1977ua,Bernstein:1985th} which have non-relativistic velocities. The velocity of the particles strongly impacts structure formation. For example, \emph{hot} DM (HDM), which is highly relativistic, is excluded as it would have wiped out all small structures in the Universe~\cite{Abazajian:2004zh,dePutter:2012sh}. But CDM may have problems, too: it possibly forms ``too many'' small halos, which might even have the ``wrong'' structure (known small scale issues include the missing satellite problem~\cite{Klypin:1999uc,Moore:1999nt}, the abundance of isolated small halos~\cite{Klypin:2014ira}, the too-big-to-fail problem~\cite{BoylanKolchin:2011de,Papastergis:2014aba}, and the cusp-core problem~\cite{Dubinski:1991bm,Moore:1994yx}). While these may be cured once baryons are correctly included in the simulations~(see, e.g.,~\cite{Schaller:2014uwa,Chan:2015tna,Henson:2016eip}), an alternative attempt is to consider non-cold DM settings. In the literature, these ideas have triggered the slightly unfortunate terminology of \emph{warm} DM (WDM), i.e., DM particles with a \emph{thermal} spectrum (i.e., Bose-Einstein or Fermi-Dirac) of a temperature close to their mass. This is also assumed in many astrophysical studies, from Lyman-$\alpha$ bounds~\cite{Narayanan:2000tp,Viel:2005qj,Boyarsky:2008xj,Viel:2013apy} over galaxy formation~\cite{Menci:2013ght,Menci:2016eww,Menci:2016eui} to $N$-body simulations~\cite{Yoshida:2003rm,Lovell:2013ola,Bose:2016irl}. However, looking at realistic settings, many DM-models feature a \emph{non-thermal} spectrum, which one cannot associate any temperature with.

Among the most popular non-cold DM candidates is a sterile neutrino with a mass of a few keV, see Ref.~\cite{Adhikari:2016bei} for a recent collection of information on this topic. Sterile neutrinos are motivated from a particle theory point of view, as they are related to light (active) neutrinos and possibly even involved in their mass generation, see e.g.\ Refs.~\cite{Ky:2005yq,Dias:2005yh,Shaposhnikov:2006nn,Cogollo:2009yi,Dias:2010vt,Lindner:2010wr,Kusenko:2010ik,Merle:2011yv,Araki:2011zg,Adulpravitchai:2011rq,Zhang:2011vh,Robinson:2012wu,Mavromatos:2012cc,Dev:2012sg,Takahashi:2013eva,Borah:2013waa,Merle:2013gea,Robinson:2014bma} for concrete models. Furthermore, the topic has gained some attention in recent years, because a sterile neutrino $N$ would -- with a very small rate -- decay like $N\to \nu \gamma$, thereby producing a nearly monoenergetic X-ray photon. A detection of the corresponding line signal has been claimed by two groups in 2014~\cite{Bulbul:2014sua,Boyarsky:2014jta}, but it was very actively disputed -- see Ref.~\cite{Adhikari:2016bei} for a detailed discussion and Ref.~\cite{Aharonian:2016gzq} for the (still ambiguous) data that the Hitomi satellite could take before enduring its unfortunate fate. It may also be worth mentioning that there could be a ``dip'' in the cluster data~\cite{Conlon:2016lxl}, tending to make a non-observation of the line in stacked data more consistent.

Sterile neutrinos with a sufficiently large lifetime are excellent DM candidates provided that 1.)~an efficient production mechanism exists in the early Universe which 2.)~produces a spectrum in accordance with cosmic structure formation. The most generic idea is to produce sterile neutrinos by their small admixtures to active neutrinos, by non-resonant active-sterile transitions, a mechanism first proposed by Langacker~\cite{Langacker:1989sv} and related to DM by Dodelson and Widrow (DW)~\cite{Dodelson:1993je}. In modern terms, one would refer to these sterile neutrinos as FIMPs (Feebly Interacting Massive Particles~\cite{Hall:2009bx}), which do not thermalise but are instead produced gradually in the early Universe via freeze-in by their feeble couplings to Standard Model (SM) particles. While this mechanism is rather simple, it is excluded by data, due to the particles being too close to the HDM limit~\cite{Seljak:2006qw,Viel:2013apy}; still, it is an unavoidable addendum to the spectrum produced by any other mechanism, which can modify the DM distribution function at least for sterile neutrino masses below $3$~keV~\cite{Merle:2015vzu}.

A popular alternative is based on a resonant enhancement of the active-sterile neutrino transitions by the presence of a primordial lepton number asymmetry. First proposed by Enqvist and collaborators~\cite{Enqvist:1990ek} and related to DM by Shi and Fuller (SF)~\cite{Shi:1998km}, this mechanism has been studied actively~\cite{Abazajian:2001nj,Laine:2008pg,Kishimoto:2008ic,Canetti:2012kh,Abazajian:2014gza,Ghiglieri:2015jua,Venumadhav:2015pla}, and it does indeed yield a spectrum colder than that produced by DW. However, while there is no perfect agreement between the results of different groups, at least for the results that are publicly available~\cite{Abazajian:2001nj,Abazajian:2014gza,Venumadhav:2015pla} it seems unclear whether they are in full agreement with cosmic structure formation~\cite{Merle:2014xpa,Horiuchi:2015qri,Schneider:2016uqi}.

On the other hand, one could produce sterile neutrinos thermally via freeze-out, if they had non-trivial charges beyond the SM, as long as the resulting overabundance is diluted by a sufficiently efficient production of additional entropy~\cite{Bezrukov:2009th,Nemevsek:2012cd,Patwardhan:2015kga}; this, however, is constrained by big bang nucleosynthesis~\cite{King:2012wg}.

A completely different direction is to produce sterile neutrinos by the decays of other particles. A generic possibility is a decaying singlet scalar $S$, which can easily couple to sterile neutrinos after having been produced itself in the first place. Apart from this scalar being the inflaton~\cite{Shaposhnikov:2006xi,Bezrukov:2009yw,Bezrukov:2014nza}, in which case it had been present all along in the early Universe, one can tune the Higgs portal coupling of $S$ in such a way that it is either produced like a WIMP via freeze-out~\cite{Kusenko:2006rh,Petraki:2007gq,Kusenko:2009up} or like a FIMP via freeze-in~\cite{Merle:2013wta,Merle:2015oja}. Variants have been presented in~\cite{Klasen:2013ypa,Kang:2014cia,Frigerio:2014ifa,Adulpravitchai:2014xna,Humbert:2015epa,Ayazi:2015jij,McDonald:2015ljz,Adulpravitchai:2015mna,Shakya:2015xnx,Kaneta:2016vkq}, some also featuring other parent particles such as vectors~\cite{Boyanovsky:2008nc,Shuve:2014doa,Biswas:2016bfo}, Dirac fermions~\cite{Abada:2014zra}, or pions~\cite{Lello:2014yha,Lello:2015uma}; related aspects such as influences from inflation~\cite{Nurmi:2015ema} or thermal corrections~\cite{Drewes:2015eoa} have been discussed, too. 

Our main goal is to close a gap in the treatment of sterile neutrino production from scalar decays. In earlier works~\cite{Kusenko:2006rh,Petraki:2007gq,Merle:2013wta,Merle:2015oja}, simplifying assumptions have been applied to at all arrive at a result, such as neglecting active-sterile mixing, taking the number of relativistic degrees of freedom to be constant, and assuming a heavy scalar. This is also true for a previous paper by two of us (AM \& MT)~\cite{Merle:2015oja}, which was the first to show how to numerically compute momentum distribution functions of sterile neutrinos from scalar decay. While it has been proven that neglecting active-sterile mixing is in fact a very good assumption~\cite{Merle:2015vzu}, going to the low-mass region of the singlet scalar -- where the number of degrees of freedom is not constant -- poses new technical challenges. The only treatment available for this regime was put forward in~\cite{Adulpravitchai:2014xna}. However, the authors only used rate equations, such that an actual computation of the momentum distribution function and the confrontation with structure formation data cannot be reliably performed. We will in this work present the full numerical computation on the level of momentum distribution functions instead, including a detailed treatment of all subtleties related to the many new technical aspects that arise in this regime. We will give an a-posteriori justification of some of the assumptions made in~\cite{Adulpravitchai:2014xna} (such as the Higgs staying equilibrated during $S$-production), while we will also reveal that others (like the Higgs degrees of freedom in the unbroken phase) are maybe less justified. We will furthermore show in great detail how to constrain the resulting spectra by cosmic structure formation.

Our study presents the first complete treatment of scalar decay production in the whole parameter space. It involves a fully general solution of the production equations on the level of distribution functions. Apart from our techniques being transferable to virtually any type of decay production, our study will guide the particle physics community towards using limits from cosmic structure formation without having to leave their comfort zone completely. We thus contribute to closing the gap between particle physics models, early DM-production, and their phenomenological consequences.

This text is structured as follows. We first present a qualitative discussion of decay production of sterile neutrinos in Sec.~\ref{sec:QualitativeDiscussion}, before explaining how to solve the evolution equations and to apply the relevant bounds in Sec.~\ref{sec:Technicalities}. Our main results are presented in Sec.~\ref{sec:Results}, which features a thorough discussion of all relevant aspects from the production process over the distribution functions to structure formation. We conclude in Sec.~\ref{sec:CandO}. Technical aspects on the Boltzmann equation, on the evolution of the Higgs distribution, on the failure of the free-streaming horizon as a reliable estimator, and on the robustness of our half-mode analysis are discussed in Appendices~\ref{app:A:DetailsComputation}, \ref{app:B:Higgs-FO}, \ref{app:C:FSvsHalfMode}, and~\ref{app:D:HalfmodeThreshold}, respectively.


